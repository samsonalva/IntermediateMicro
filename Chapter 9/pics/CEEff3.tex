\documentclass{article}
\usepackage{epsfig}
\usepackage[dvips]{color}
\setlength{\textwidth}{100 cm}
\setlength{\textheight}{80 cm}
\begin{document}
\pagestyle{empty}
\documentclass{article}
\usepackage{epsfig}
\usepackage[dvips]{color}
\setlength{\textwidth}{100 cm}
\setlength{\textheight}{80 cm}
\begin{document}
\pagestyle{empty}
\documentclass{article}
\usepackage{epsfig}
\usepackage[dvips]{color}
\setlength{\textwidth}{100 cm}
\setlength{\textheight}{80 cm}
\begin{document}
\pagestyle{empty}
\documentclass{article}
\usepackage{epsfig}
\usepackage[dvips]{color}
\setlength{\textwidth}{100 cm}
\setlength{\textheight}{80 cm}
\begin{document}
\pagestyle{empty}
\input{CEEff3.pstex_t}
\end{document}
\end{document}
\end{document}
\end{document}